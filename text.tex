\documentclass[12pt,letter]{article}

\renewcommand{\textfraction}{0.15}
\renewcommand{\topfraction}{0.85}
\renewcommand{\bottomfraction}{0.65}
\renewcommand{\floatpagefraction}{0.60}
%\usepackage[small, sf]{titlesec}
%\usepackage[lined,boxed]{algorithm2e}
\usepackage{amssymb}
\usepackage{amsmath}
\usepackage{graphicx}
\usepackage{fancyhdr}
\usepackage{alltt}

\usepackage[letterpaper,textwidth=6.5in,textheight=8.5in,top=1.3in,left=1in,right=1in,bottom=1.3in]{geometry}
\usepackage{longtable}

\pagestyle{fancy}
%\rhead{Fang-Yu Rao 00222 05304\\
%Fang-Yu Rao 00222 05304}
\lhead{CS182 Spring 2013 Midterm Solution and Rubric}
\rhead{Prof. Alex Pothen and Vernon Rego}
\renewcommand{\headrulewidth}{0pt}

%\textheight=8in

\linespread{1}

\title{}
%\author{Fang-Yu Rao \\
%\texttt{raof@purdue.edu}
%}
%\date{\today}

\begin{document}
%\maketitle

\begin{enumerate}

%p1
\item 

%p2
\item 

%p3
\item

%p4
\item 

%p5 
\item Grader: Fang-Yu Rao
	\begin{enumerate}
	\item If $A = \{1, 2, 3\}$, $B = \{3, 4, 5\}$, and $C = \{ 1, 3, 4\}$, then 
	we can see $(A \cup C) \subset (A \cup B)$ but 
	$C \not\subset B$.\\
	{\bf Note}:
	\begin{itemize}
	\item 0 pt will be given if nothing is written down on the paper.\\
	\item 1 pt will be given if the counterexample given is wrong.\\
	\item 4 pts will be given if the answer is correct.
	\end{itemize}

	\item To prove $C \subset B$, we need to prove that 
	for all $x \in C$, $x$ is also in $B$.
	\par
	Let $x \in C$, then it is true that ``$x \in A$ or $x \in C$," i.e., $x \in (A \cup C)$.
	Because we are given that $(A \cup C) \subset (A \cup B)$, 
	$x \in (A \cup B)$, i.e., $x \in A$ or $x \in B$.

	\par
	If $x \in B$, then we are done.
	Consider the case when $x \in A$. 
	Then we know it is true that
	$x \in A$ and $x \in C$, which implies $x \in (A \cap C)$.
	By the given condition that $(A \cap C) \subset (A \cap B)$, 
	we have $x \in (A \cap B)$, i.e., it is true that 
	``$x \in A$ and $x \in B$".

	\par
	In either case, $x \in B$.

	{\bf Note}:
	\begin{itemize}
	\item 0 pt will be given if nothing is written down on the paper.\\
	\item 1 pt will be given if the argument as a whole is incorrect.\\
	\item 2 pts will be given if you are able to prove the implication that $(x \in C) \rightarrow (x \in A \mbox { or } x \in B)$.
	\item Another 1 pt will be given if the case when $x \in B$ is discussed or mentioned.
	\item Another 3 pts will be given if the case when $x \in A$ is discussed, i.e.,  
	you are able to prove that $(x \in C) \rightarrow (x \in B)$ when $x \in A$.
	\item Since Venn diagram is allowed in this problem, 6pts will be given if the constructed Venn diagram makes sense.
	\item Generally, if the notations are used incorrectly, then the answer would fall in the second case, i.e., 
	it will be considered incorrect as a whole. 
	For example, something like the following would be considered incorrect:
	\[
		[(x \in A) \mbox{ or } (x \in C)]  \subset [(x \in A) \mbox{ or } (x \in B)].
	\]
	\end{itemize}
	\end{enumerate}

%p6
\item Grader: Ravi Kiran Rao Bukka	
	\begin{enumerate}
	\item	0 pt for wrong answer and 1 pt for correct answer. Answers are False, True, False, False for i,ii,iii,iv respectively.
	\item	Proving R is reflexive and an equivalence relation. The definitions of reflexive, symmetric and transitive are as per the text book.
	\begin{itemize}
	\item 1 pt for explaining symmetric property seperately or as part of the proof.
	\item 1 pt for explaining transitive property seperately or as part of the proof.
	\item 3 pt for proving R is reflexive especially when b is not equal to a. If b is equal to a then $(a,a) \in R$ for all a. Hence reflexive. 
		If b is not equal to a then:
		If $(a,b) \in R$ from symmetric property $(b,a) \in R$. Now, from transitive property if $(a,b),(b,a) \in R$ then $(a,a) \in R$ by definition. 
		Hence R is reflexive.
	\item 1 pt for proving R is an equivalence relation. Just a line stating that R is reflexive, symmetric and transitive and hence an equivalence relation.
	\end{itemize}
	\end{enumerate}

%p7
\item 

%p8
\item 

\item 

%p10
\item 


\end{enumerate}

\end{document}

